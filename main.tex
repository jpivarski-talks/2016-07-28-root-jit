\documentclass{beamer}

%
% Choose how your presentation looks.
%
% For more themes, color themes and font themes, see:
% http://deic.uab.es/~iblanes/beamer_gallery/index_by_theme.html
%
\mode<presentation>
{
  \usetheme{default}      % or try Darmstadt, Madrid, Warsaw, ...
  \usecolortheme{default} % or try albatross, beaver, crane, ...
  \usefonttheme{default}  % or try serif, structurebold, ...
  \setbeamertemplate{navigation symbols}{}
  \setbeamertemplate{caption}[numbered]
  \setbeamertemplate{footline}[page number]
  \setbeamercolor{frametitle}{fg=white}
  \setbeamercolor{footline}{fg=black}
} 

\usepackage[english]{babel}
\usepackage[utf8x]{inputenc}
\usepackage{tikz}
\usepackage{listings}
\usepackage{courier}

\xdefinecolor{darkblue}{rgb}{0.1,0.1,0.7}
\xdefinecolor{dianablue}{rgb}{0.18,0.24,0.31}
\definecolor{commentgreen}{rgb}{0,0.6,0}
\definecolor{stringmauve}{rgb}{0.58,0,0.82}

\lstset{ %
  backgroundcolor=\color{white},      % choose the background color
  basicstyle=\ttfamily\scriptsize,         % size of fonts used for the code
  breaklines=true,                    % automatic line breaking only at whitespace
  captionpos=b,                       % sets the caption-position to bottom
  commentstyle=\color{commentgreen},  % comment style
  escapeinside={\%*}{*)},             % if you want to add LaTeX within your code
  keywordstyle=\color{blue},          % keyword style
  stringstyle=\color{stringmauve},    % string literal style
  showstringspaces=false,
  showlines=true
}

\lstdefinelanguage{scala}{
  morekeywords={abstract,case,catch,class,def,%
    do,else,extends,false,final,finally,%
    for,if,implicit,import,match,mixin,%
    new,null,object,override,package,%
    private,protected,requires,return,sealed,%
    super,this,throw,trait,true,try,%
    type,val,var,while,with,yield},
  otherkeywords={=>,<-,<\%,<:,>:,\#,@},
  sensitive=true,
  morecomment=[l]{//},
  morecomment=[n]{/*}{*/},
  morestring=[b]",
  morestring=[b]',
  morestring=[b]"""
}

\lstdefinelanguage{cpp}{
  morekeywords={for,if,class,public,typedef,struct,double,return,Int_t,int,void,Long64_t},
  otherkeywords={=>,<-,<\%,<:,>:,\#,@},
  sensitive=true,
  morecomment=[l]{//},
  morecomment=[n]{/*}{*/},
  morestring=[b]",
  morestring=[b]',
  morestring=[b]"""
}

\title[2016-07-28-root-jit]{Histogram abstraction as fast as C++}
\author{Jim Pivarski}
\institute{Princeton University -- DIANA}
\date{July 28, 2016}

\begin{document}

\logo{\pgfputat{\pgfxy(0.11, 8)}{\pgfbox[right,base]{\tikz{\filldraw[fill=dianablue, draw=none] (0 cm, 0 cm) rectangle (50 cm, 1 cm);}}}\pgfputat{\pgfxy(0.11, -0.6)}{\pgfbox[right,base]{\tikz{\filldraw[fill=dianablue, draw=none] (0 cm, 0 cm) rectangle (50 cm, 1 cm);}\includegraphics[height=0.99 cm]{diana-hep-logo.png}\tikz{\filldraw[fill=dianablue, draw=none] (0 cm, 0 cm) rectangle (4.9 cm, 1 cm);}}}}

\begin{frame}
  \titlepage
\end{frame}

\logo{\pgfputat{\pgfxy(0.11, 8)}{\pgfbox[right,base]{\tikz{\filldraw[fill=dianablue, draw=none] (0 cm, 0 cm) rectangle (50 cm, 1 cm);}\includegraphics[height=1 cm]{diana-hep-logo.png}}}}

% Uncomment these lines for an automatically generated outline.
%\begin{frame}{Outline}
%  \tableofcontents
%\end{frame}

\begin{frame}{Start with the punchline}
\vspace{0.25 cm}
Wall time to fill 100-bin histogram with {\tt \small 2*x}, selecting on {\tt \small !y}.

TTree has 1 million entries, fully loaded into TTreeCache.

\renewcommand{\arraystretch}{1.2}
\vspace{0.25 cm}
\begin{tabular}{l c c c c}
\textcolor{darkblue}{ROOT} & & & & \\
Fill method & Preparation & Prep (ms) & Run (ms) & Relative \\\hline
C++ & compilation & 12.6 & \textcolor{red}{198} & \textcolor{red}{1} \\
PyROOT & none & & 15,700 & 79 \\
TFormula & first pass & 380 & 317 & 1.6\vspace{0.25 cm} \\
\textcolor{darkblue}{Histogrammar} & & & & \\
Fill method & Preparation & Prep (ms) & Run (ms) & Relative \\\hline
pure Python & none & & 21,000 & 105 \\
with Numpy & first pass & 814 & 312 & 1.6 \\
with ROOT & JIT-compilation & 219 & \textcolor{red}{196} & \textcolor{red}{0.98} \\
\end{tabular}
\end{frame}

\begin{frame}[fragile]{ROOT code in the test}

\vspace{0.3 cm}
\hfill \textcolor{darkblue}{C++ ROOT}

\vspace{-0.5 cm}
\begin{minipage}{0.8\linewidth}
\begin{lstlisting}[language=cpp]
TH1D control1("control1", "", 100, -10, 10);
// start stopwatch
ttree->SetBranchAddress("x", &input_x);
ttree->SetBranchAddress("y", &input_y);
for (i = 0;  i < ttreeEntries;  ++i) {
  root_ttree->GetEntry(i);
  if (!input_y)
    control1.Fill(2 * input_x);
}
// end stopwatch
\end{lstlisting}
\end{minipage}

\hfill \textcolor{darkblue}{PyROOT}

\vspace{-0.5 cm}
\begin{minipage}{0.8\linewidth}
\begin{lstlisting}[language=python]
control2 = ROOT.TH1D("control2", "", 100, -10, 10)
# start stopwatch
for entry in root_ttree:
    if not entry.y:
        control2.Fill(2 * entry.x)
# end stopwatch
\end{lstlisting}
\end{minipage}

\hfill \textcolor{darkblue}{TFormula}

\vspace{-0.5 cm}
\begin{minipage}{0.8\linewidth}
\begin{lstlisting}[language=python]
control3 = ROOT.TH1D("control3", "", 100, -10, 10)
# start stopwatch
root_ttree.Draw("2 * x >>+ control3", "!y", "goff")
# end stopwatch
\end{lstlisting}
\end{minipage}
\end{frame}

\begin{frame}[fragile]{Histogrammar code in the test}
\vspace{0.3 cm}

\hfill \textcolor{darkblue}{pure Python}

\vspace{-0.5 cm}
\begin{minipage}{0.8\linewidth}
\begin{lstlisting}[language=python]
experiment1 = Select(lambda t: not t.y,
                  Bin(100, -10, 10,
                      lambda t: 2 * t.x))
# start stopwatch
for t in list_of_entries:
    experiment1.fill(t)
# end stopwatch
\end{lstlisting}
\end{minipage}

\vspace{0.5 cm}
\hfill \textcolor{darkblue}{with Numpy}

\vspace{-0.5 cm}
\begin{minipage}{0.8\linewidth}
\begin{lstlisting}[language=python]
experiment2 = Select("logical_not(y)",
\end{lstlisting}
\end{minipage}

\vspace{-0.45 cm}
\begin{minipage}{\linewidth}
\begin{lstlisting}[language=python]
                  Bin(100, -10, 10, "2 * x"))
# start stopwatch
experiment2.numpy(dict_of_numpy_arrays)
# end stopwatch
\end{lstlisting}
\end{minipage}

\vspace{0.5 cm}
\hfill \textcolor{darkblue}{with ROOT}

\vspace{-0.5 cm}
\begin{minipage}{0.8\linewidth}
\begin{lstlisting}[language=python]
experiment3 = Select("!y",
\end{lstlisting}
\end{minipage}

\vspace{-0.45 cm}
\begin{minipage}{\linewidth}
\begin{lstlisting}[language=python]
                  Bin(100, -10, 10, "2 * x"))
# start stopwatch
experiment3.cling(root_ttree)
# end stopwatch
\end{lstlisting}
\end{minipage}
\end{frame}

\begin{frame}{Dissect the differences}
\vspace{0.5 cm}
\renewcommand{\arraystretch}{1.2}
\begin{tabular}{l c}
ROOT C++ with TTree read and histogram fill & 198 ms \\
ROOT C++ filling histogram with zeros & 25 ms \\
C++ empty for loop & 5 ms \\
 & \\
Histogrammar-Cling with TTree read and histogram fill & 196 ms \\
Histogrammar-Cling filling histogram with zeros & 19 ms \\
C++ empty for loop & 5 ms \\
\end{tabular}

\vspace{1 cm}
ROOT C++ and Histogrammar-Cling are equivalent because they both call {\tt\small ttree->GetEntry(i)}, which takes 90\% of the time.
\end{frame}

\begin{frame}{}
\begin{center}
\textcolor{darkblue}{\Large Back to the beginning for some context}
\end{center}
\end{frame}

\begin{frame}[fragile]{Grammar of histograms}
\vspace{0.5 cm}
\begin{center}
\includegraphics[width=0.5\linewidth]{histogrammar-logo.png}
\end{center}

\textcolor{darkblue}{A suite of classes that each aggregate a simple thing and can aggregate complex things when combined.}

\vspace{0.35 cm}
For instance, an ordinary histogram:

\begin{lstlisting}[language=python]
histogram = Bin(10, -5, 5, "pt", Count())
\end{lstlisting}

and a histogram with {\tt\small Sumw2}:

\begin{lstlisting}[language=python]
histogram = Bin(10, -5, 5, "pt", Branch(Count(), Count("x**2")))
\end{lstlisting}

The content in each bin is now a count {\it and} a sum of weights squared.

\begin{center}
\textcolor{darkblue}{\url{http://histogrammar.org}}
\end{center}
\end{frame}

\begin{frame}{Catalog of primitives}

\vspace{0.5 cm}
\begin{columns}
\column{0.45\linewidth}
\textcolor{darkblue}{\bf Count:} sum of weights \\
\textcolor{darkblue}{\bf Sum:} sum of anything \\
\textcolor{darkblue}{\bf Average:} average \\
\textcolor{darkblue}{\bf Deviate:} average and variance (for a profile plot) \\
\textcolor{darkblue}{\bf Minimize:} minimum value \\
\textcolor{darkblue}{\bf Maximize:} maximum value \\
\textcolor{darkblue}{\bf Fraction:} ratio of contents (for an efficiency plot) \\
\textcolor{darkblue}{\bf Stack:} cumulative filling \\
\textcolor{darkblue}{\bf Select:} apply a cut \\
\textcolor{darkblue}{\bf Limit:} drop detail when large

\column{0.6\linewidth}
\textcolor{darkblue}{\bf Bin:} regular binning (normal histogram) \\
\textcolor{darkblue}{\bf SparselyBin:} bins in a hashmap \\
\textcolor{darkblue}{\bf CentrallyBin:} irregular bins defined by the centers of each bin \\
\textcolor{darkblue}{\bf IrregularlyBin:} \ldots defined by edges \\
\textcolor{darkblue}{\bf Categorize:} bin unique by string value (bar chart as a kind of histogram) \\
\textcolor{darkblue}{\bf Label:} directory with string-based keys \\
\textcolor{darkblue}{\bf UntypedLabel:} no type restrictions \\
\textcolor{darkblue}{\bf Index:} homogeneous list \\
\textcolor{darkblue}{\bf Branch:} heterogeneous tuple \\
\textcolor{darkblue}{\bf Bag:} accumulate raw data (scatter) \\
\end{columns}

\vspace{0.25 cm}
\begin{itemize}
\item All are one-pass, commutative, and associative (parallelizable).
\item Implement in Python, Scala, \textcolor{gray}{C++, Julia,} \textcolor{lightgray}{R, Javascript\ldots}
\item Many filling back-ends, plotting front-ends, including ROOT.
\end{itemize}
\end{frame}

\begin{frame}[fragile]{Using ROOT as a back-end}
When a user types

\begin{center}
\begin{minipage}{0.9\linewidth}
\begin{lstlisting}[language=python, basicstyle=\ttfamily\small]
h = Select("!y", Bin(100, -10, 10, "2 * x"))
h.cling(root_ttree)
\end{lstlisting}
\end{minipage}
\end{center}

Histogrammar-Python generates C++ code

\begin{itemize}
\item for this specific combination ({\tt\small Select-Bin-Count})
\item for this specific TTree
\item with these derived fields
\end{itemize}

and sends it to Cling to compile and run.
\end{frame}

\begin{frame}[fragile]{The code generated in this example}
\vspace{0.5 cm}
\begin{columns}
\column{0.35\linewidth}
\begin{lstlisting}[language=cpp, basicstyle=\ttfamily\fontsize{5}{4}\selectfont]
class HistogrammarClingFiller_0 {
public:
  typedef struct {
    double entries;
    double underflow;
    double overflow;
    double nanflow;
    double values[100];
    double& getValues(int i) { return values[i]; }
  } Bn100CtCtCtCt;

  typedef struct {
    double entries;
    Bn100CtCtCtCt cut;
  } SeBn100CtCtCtCt;

  double weight_0;
  double weight_1;
  Int_t input_y;
  double input_x;
  double quantity_0;
  double quantity_1;
  int bin_0;
  SeBn100CtCtCtCt storage;

  void fillall(TTree* ttree, Long64_t start, Long64_t end) {
    storage.entries = 0.0;
    storage.cut.entries = 0.0;
\end{lstlisting}

\column{0.35\linewidth}
\begin{lstlisting}[language=cpp, basicstyle=\ttfamily\fontsize{5}{4}\selectfont]
    storage.cut.nanflow = 0.0;
    storage.cut.underflow = 0.0;
    storage.cut.overflow = 0.0;
    for (bin_0 = 0;  bin_0 < 100;  ++bin_0) {
      storage.cut.values[bin_0] = 0.0;
    }
    weight_0 = 1.0;
    ttree->SetBranchAddress("x", &input_x);
    ttree->SetBranchAddress("y", &input_y);

    if (start < 0) start = 0;
    if (end < 0) end = ttree->GetEntries();
    for (;  start < end;  ++start) {
      ttree->GetEntry(start);
      quantity_0 = !input_y;
      quantity_1 = 2 * input_x;
      storage.entries += weight_0;
      if (!isnan(quantity_0)  &&  quantity_0 > 0.0) {
        weight_1 = weight_0 * quantity_0;
\end{lstlisting}

\column{0.35\linewidth}
\begin{lstlisting}[language=cpp, basicstyle=\ttfamily\fontsize{5}{4}\selectfont]
        storage.cut.entries += weight_1;
        if (isnan(quantity_1)) {
          storage.cut.nanflow += weight_1;
        }
        else if (quantity_1 < -10.0) {
          storage.cut.underflow += weight_1;
        }
        else if (quantity_1 >= 10.0) {
          storage.cut.overflow += weight_1;
        }
        else {
          bin_0 = floor((quantity_1 - -10.0) * 5.0);
          storage.cut.values[bin_0] += weight_1;
        }
      }
    }

    ttree->ResetBranchAddresses();
  }
};
\end{lstlisting}
\end{columns}
\end{frame}

\begin{frame}{Properties of this code}

\begin{itemize}\setlength{\itemsep}{0.35 cm}
\item Arbitrarily deep nesting: any aggregator that can be constructed out of primitives generates working C++ code.
\item Human-readable for debugging, with good indentation.
\item All but three primitives ({\tt\small SparselyBin}, {\tt\small Categorize}, {\tt\small Bag}) generate memory-contiguous code, and nothing is allocated in the loop (not even on the stack). It should use a CPU cache efficiently.
\item Repeated expressions are computed once.
\item Expressions are sanitized, with code-transformations applied to the abstract syntax tree (AST), not raw strings.
\end{itemize}
\end{frame}

\begin{frame}{What this means for a }






\end{frame}





\begin{frame}[fragile]{Analysis code must be declarative to use it}
\vspace{0.1 cm}
\begin{lstlisting}[language=python]
from ROOT import *; from histogrammar import *             
gInterpreter.AddIncludePath("Event.h")         # works for
gInterpreter.ProcessLine(".L Event.cxx")       # complex events
tfileData = TFile("data.root"); data = tfileEvent.Get("Events")
tfileMC1 = TFile("mc1.root"); mc1 = tfileEvent.Get("Events")
tfileMC2 = TFile("mc2.root"); mc2 = tfileEvent.Get("Events")

plots = UntypedLabel(           # book all aggregators together
    mass = Bin(100, 0, 50, "event.dijet.mass()"),  # quoted C++
    pt =   Bin(100, 0, 300, "event.dijet.pt()"),      # hist
    profile = Bin(100, 0, 300, "event.dijet.pt()",    # prof
                  Deviate("event.dijet.eta()")),
    twodim =  Bin(100, 0, 300, "event.dijet.pt()",    # 2D
                  Bin(100, -5, 5, "event.dijet.eta()")),
    trigeff = Fraction("trigger",                     # eff
                  Bin(100, 0, 300, "event.maxjet.pt()")))

plotsData = plots.copy().cling(data)   # fill all at once
plotsMC1 = plots.copy().cling(mc1)     # for each sample
plotsMC2 = plots.copy().cling(mc2)

tefficiency = plotsData.get("trigeff").root("efficiency vs pt")
tefficiency.Draw()
tstack = Stack.build(plotsMC1.get("mass"), plotsMC2.get("mass"))
tstack.root("backgrounds").Draw()      # get ordinary ROOT plots
\end{lstlisting}
\end{frame}

\begin{frame}{}
The fact that the analyst's histogramming code is not imperative is what makes it {\it possible} to fill in different ways.
\end{frame}

\begin{frame}{See more at http://histogrammar.org}
\vspace{0.5 cm}

\textcolor{darkblue}{Currently available:}
\begin{itemize}
\item Distributed filling, e.g. across a Spark cluster (JVM/Python)
\item Spark-SQL enhancement
\item Numpy acceleration: for Scikit-Learn, etc.
\item ROOT-Cling acceleration: already valuable to ordinary physicists doing ordinary analyses
\end{itemize}

\vfill
\textcolor{darkblue}{Forseen/in progress:}
\begin{itemize}
\item Julia (a language designed around JIT-compilation)
\item GPU (me) and FPGA (Peter Hansen): for debugging workflows in remote environments, like trigger development
\item JIT-Java code, R, Javascript: for industry
\end{itemize}

\vspace{0.25 cm}
All languages use the same suite of 20 aggregation primitives.
\end{frame}

\begin{frame}{Proposal/request}
When a Histogrammar version 1.0 is reached, can we bundle it in the PyROOT distribution?

\begin{uncoverenv}<2->
\vspace{0.5 cm}
It has no explicit dependencies (e.g. Numpy will work if Numpy is available, doesn't complain if it's not).

\vspace{0.5 cm}
Continually tested in Pythons 2.6 (SL6), 2.7, and 3.4.

\vspace{0.5 cm}
It should be made to interact well with the new ``chains of functional primitives'' and parallel TTree reading.
\end{uncoverenv}
\end{frame}

\begin{frame}[fragile]{Backup: evidence that TTreeCache is filled}
\scriptsize
\begin{lstlisting}
TFile::PrintCacheStats()

******TreeCache statistics for tree: big in file: test/big.root ******
Number of branches in the cache ...: 2
Cache Efficiency ..................: 1.000000
Cache Efficiency Rel...............: 1.000000
Learn entries......................: 100
Cached Reading.....................: 8022693 bytes in 1 transactions
Reading............................: 0 bytes in 0 uncached transactions
Readahead..........................: 256000 bytes with overhead = 0 bytes
Average transaction................: 8022.693000 Kbytes
Number of blocks in current cache..: 377, total size: 8022693

These values were constant throughout the test:
    TFile::GetBytesRead() = 8031424
    TFile::GetReadCalls() = 5
\end{lstlisting}

\small So {\tt\small ttree->GetEntry(i)} isn't the bottleneck due to disk I/O.
\end{frame}

\end{document}

%% TFile::PrintCacheStats()

%% ******TreeCache statistics for tree: big in file: test/big.root ******
%% Number of branches in the cache ...: 2
%% Cache Efficiency ..................: 1.000000
%% Cache Efficiency Rel...............: 1.000000
%% Learn entries......................: 100
%% Cached Reading.....................: 8022693 bytes in 1 transactions
%% Reading............................: 0 bytes in 0 uncached transactions
%% Readahead..........................: 256000 bytes with overhead = 0 bytes
%% Average transaction................: 8022.693000 Kbytes
%% Number of blocks in current cache..: 377, total size: 8022693

%% These values were constant throughout the test:
%%     TFile::GetBytesRead() = 8031424
%%     TFile::GetReadCalls() = 5

%% line |
%%    1 | class HistogrammarClingFiller_0 {
%%    2 | public:
%%    3 |
%%    4 |   typedef struct {
%%    5 |     double entries;
%%    6 |     double underflow;
%%    7 |     double overflow;
%%    8 |     double nanflow;
%%    9 |     double values[100];
%%   10 |     double& getValues(int i) { return values[i]; }
%%   11 |   } Bn100CtCtCtCt;
%%   12 |
%%   13 |   typedef struct {
%%   14 |     double entries;
%%   15 |     Bn100CtCtCtCt cut;
%%   16 |   } SeBn100CtCtCtCt;
%%   17 |
%%   18 |   double weight_0;
%%   19 |   double weight_1;
%%   20 |   Int_t input_boolean;
%%   21 |   double input_noholes;
%%   22 |
%%   23 |   double quantity_0;
%%   24 |   double quantity_1;
%%   25 |   int bin_0;
%%   26 |   SeBn100CtCtCtCt storage;
%%   27 |
%%   28 |   bool isnan(double x) { return x != x; }
%%   29 |   bool isnan(float x) { return x != x; }
%%   30 |   bool isnan(int x) { return false; }
%%   31 |
%%   32 |   void fillall(TTree* ttree, Long64_t start, Long64_t end) {
%%   33 |     storage.entries = 0.0;
%%   34 |     storage.cut.entries = 0.0;
%%   35 |     storage.cut.nanflow = 0.0;
%%   36 |     storage.cut.underflow = 0.0;
%%   37 |     storage.cut.overflow = 0.0;
%%   38 |     for (bin_0 = 0;  bin_0 < 100;  ++bin_0) {
%%   39 |       storage.cut.values[bin_0] = 0.0;
%%   40 |     }
%%   41 |     weight_0 = 1.0;
%%   42 |     ttree->SetBranchAddress("noholes", &input_noholes);
%%   43 |     ttree->SetBranchAddress("boolean", &input_boolean);
%%   44 |
%%   45 |     if (start < 0) start = 0;
%%   46 |     if (end < 0) end = ttree->GetEntries();
%%   47 |     for (;  start < end;  ++start) {
%%   48 |       ttree->GetEntry(start);
%%   49 |       quantity_0 = !input_boolean;
%%   50 |       quantity_1 = 2 * input_noholes;
%%   51 |       storage.entries += weight_0;
%%   52 |       if (!isnan(quantity_0)  &&  quantity_0 > 0.0) {
%%   53 |         weight_1 = weight_0 * quantity_0;
%%   54 |         storage.cut.entries += weight_1;
%%   55 |         if (isnan(quantity_1)) {
%%   56 |           storage.cut.nanflow += weight_1;
%%   57 |         }
%%   58 |         else if (quantity_1 < -10.0) {
%%   59 |           storage.cut.underflow += weight_1;
%%   60 |         }
%%   61 |         else if (quantity_1 >= 10.0) {
%%   62 |           storage.cut.overflow += weight_1;
%%   63 |         }
%%   64 |         else {
%%   65 |           bin_0 = floor((quantity_1 - -10.0) * 5.0);
%%   66 |           storage.cut.values[bin_0] += weight_1;
%%   67 |         }
%%   68 |       }
%%   69 |     }
%%   70 |
%%   71 |     ttree->ResetBranchAddresses();
%%   72 |   }
%%   73 | };

%% class ControlTest {
%% public:
%%   TH1D histogram;
%% 
%%   ControlTest() : histogram("control1", "", 100, -10, 10) {}
%% 
%%   Int_t input_boolean;
%%   double input_noholes;
%% 
%%   void fillall(TTree* ttree, Long64_t start, Long64_t end) {
%%     if (start < 0) start = 0;
%%     if (end < 0) end = ttree->GetEntries();
%% 
%%     ttree->SetBranchAddress("noholes", &input_noholes);
%%     ttree->SetBranchAddress("boolean", &input_boolean);
%% 
%%     for (;  start < end;  ++start) {
%%       ttree->GetEntry(start);
%% 
%%       if (!input_boolean)
%%         histogram.Fill(2 * input_noholes);
%%     }
%% 
%%     ttree->ResetBranchAddresses();
%%   }
%% };

%% histogram = ROOT.TH1D("control2", "", 100, -10, 10)
%% for row in TestRootCling.ttreeBig:
%%     if not row.boolean:
%%         histogram.Fill(2 * row.noholes)

%% Histogrammar JIT-compilation
%% 0.21904706955 -> 1.1003578233905158

%% Histogrammar running
%% 0.196442842484 -> 0.9868080820273187
%% 0.195878982544 -> 0.9839755963083809
%% 0.19778585434 -> 0.9935545480069445

%% Histogrammar running without ttree->GetEntry(i) (fill with zeros)
%% 0.0193800926208 -> 0.09735367187125231
%% 0.0191719532013 -> 0.09630810737649194
%% 0.019119977951 -> 0.09604701566954614

%% Completely empty loop (in the Histogrammar test)
%% 0.00551819801331 -> 0.02772003463656316
%% 0.00556516647339 -> 0.027955975307250087
%% 0.00550293922424 -> 0.027643383860256804

%% ROOT C++ compilation
%% 0.0126299858093 -> 0.06344528471951177

%% ROOT C++ first event
%% 0.0141220092773 -> 0.07094029343644556

%% ROOT C++ subsequent
%% 0.198187828064 -> 0.995573816892438
%% 0.19960808754 -> 1.0027083274287882
%% 0.199410915375 -> 1.0017178556787742

%% ROOT C++ subsequent without ttree->GetEntry(i) (fill with zeros)
%% 0.0239660739899 -> 0.12039082314553802
%% 0.0246810913086 -> 0.12398263061462476
%% 0.0245039463043 -> 0.1230927630087422

%% Completely empty loop (in the ROOT C++ test)
%% 0.00520300865173 -> 0.026136717039224343
%% 0.00507307052612 -> 0.02548398777256311
%% 0.00515604019165 -> 0.025900776368537416

%% PyROOT first event
%% 0.00721597671509 -> 0.03624863116482206

%% PyROOT subsequent
%% 15.7434358597 -> 79.08534385260707
%% 15.7216258049 -> 78.97578354450955
%% 15.6935698986 -> 78.83484791796619

%% TFormula first pass
%% 0.379606962204 -> 1.9069120236704813

%% TFormula subsequent
%% 0.317915201187 -> 1.597010539878666
%% 0.316998958588 -> 1.5924079002998555
%% 0.316241025925 -> 1.588600512522203

%% Numpy first
%% 0.814823865891 -> 4.093174208449328

%% Numpy subsequent
%% 0.309661865234 -> 1.5555508535947518
%% 0.312314987183 -> 1.5688785073222755
%% 0.313981056213 -> 1.5772478139522859

%% Native Histogrammar

%% Native first
%% 20.7954871655 -> 104.4637439833533

%% Native subsequent
%% 21.058177948 -> 105.78334099165966
%% 20.9882600307 -> 105.43211635553794
%% 20.9792170525 -> 105.38668999201907
